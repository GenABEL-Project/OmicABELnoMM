\chapter{Preparing input data}

\section{Overview}

\oanomm uses the filevector (also called DatABEL) file format for the
input files. Filevector files use less storage space than plain text
files and stores data in a way that make the columns and rows easily
and quickly addressable, and thus helps the computations to be done
faster\footnote{The source code of the filevector libary
  \lstinline{fvlib} can be found on Github:
  \url{https://github.com/GenABEL-Project/filevector}. This repository
  also contains a few command line utilities for manipulate filevector
  files (see \S)}. DatABEL is an R package providing an easy way to work
with filevector files\footnote{The DatABEL page on the GenABEL project
  website is \url{http://www.genabel.org/packages/DatABEL}. The R
  package can be downloaded from CRAN:
  \url{https://cran.r-project.org/web/packages/DatABEL/}.}.


Original source files can be in any format as long as there is a way
to load them into R for a table(matrix) format. Once in table format,
they can be just transformed to DatABEL format to be used by OmicABEL.


\section{Using DatABEL to convert input files to filevector format}
Start R, then use

library(DatABEL); help("DatABEL-package")\\
More info: http://www.genabel.org/packages/DatABEL\\
Start R and load DatABEL

\begin{lstlisting}[escapechar=\%]
library(DatABEL)
\end{lstlisting}

\subsection{Covariates}

The following example code shows how to artificially create covariates:
\begin{lstlisting}[escapechar=\%]
#START_FAKE_DATA
n = 2000                 # number of individuals
l = 3                    # number of covariates+1 for intercept
r = 2                    # how many columns per SNP
m = r*100000             # number of snps
t = 10000                # number of traits
set.seed(1001)
runif(3)
XL <- matrix(rnorm((l+1)*n), ncol=(l+1)) # first column should be ones (intercept)
for (i in 1:(n * (l+1))) {
    if (sample(1:100, 1) > 95){
        XL[i]=0/0    # fill in NANs
    }
}
#END_FAKE_DATA

# From here on if you have your real data stored in the matrix
# variable XL you are ok.
# How to get your data into XL depends on your original files and
# how they were stored.

# The first column of covariates has to have 1's! it is the intercept
# Make sure you add this column of ones and that you have the space for it
# without loosing your own data.
for (i in 1:n) {
   XL[i] = 1
}

# Add your own idnames!
colnames(XL) <- c("intercept", paste("cov", 1:l, sep=""))
rownames(XL) <- paste("ind", 1:n, sep="")

# Convert to databel (i.e. store the data in a file)
XL_db <- matrix2databel(XL,filename="XL",type="FLOAT")

#XL[1:n,1:(l+1)]
#XL
%
\end{lstlisting}

\subsection{Independent Variables, SNPs,CPG Sites,Measurements used to explain other Measurements}
\begin{lstlisting}[escapechar=\%]
#START_FAKE_DATA
n = 2000                 # number of individuals
l = 3                    # number of covariates+1 for intercept
r = 2                    # how many columns per SNP
m = r*100000             # number of snps
t = 10000                # number of traits
#r=2
XR <- matrix(rnorm(m*n),ncol=m)

#Assumes that you had the previous Y still stored, this will create XR linearly dependent on the Y
for(i in 1 + r*(0:((m-2)/r)) )
{
        #print(i)
        yIdx=ceiling(i/r)
        #print(i)
        #print(yIdx)
        for(j in 1:n)
        {
                XR[j,i]=Y[j,yIdx]
                for(k in 1:l)
                {
                        XR[j,i]=XR[j,i]-XL[j,k]*0.01
                }
                for(k in 1:(r-1))
                {
                        XR[j,i]=XR[j,i]-XR[j,i+k]*0.01
                }
                #XR[j,i]=XR[j,i]/2.8888
                #XR[j,i] = XR[j,i]*runif(1, 1.0-var, 1.0)

        }
}

#add missing data
for(i in 1:(n*m)){ if(sample(1:100,1) > 90) XR[i]=0/0}
#END_FAKE_DATA

#FROM here on if you have your real data stored in the matrix variable XL you are ok.
#how to get your data into XL depends on your original files and how they were stored.

#The first column of covariates has to have 1's! it is the intercepts
#Make sure you add this column of ones and that you have the space for it
#without loosing your own data.

#add your own idnames!
colnames(XR) <- paste("miss",1:m,sep="")
for(i in 1:(m/r))
{
        for(j in 1:r)
        {
                colnames(XR)[(i-1)*r+(j)] = paste0("snp",paste(i,j,sep="_") )
        }
}

#add your own idnames!
rownames(XR) <- paste("ind",1:n,sep="")

#transform to databel (store it)
XR_db <- matrix2databel(XR,filename="XR",type="FLOAT")
%
\end{lstlisting}

\subsection{Dependent Variable, Phenotypes,Measurements to be explained}

\begin{lstlisting}[escapechar=\%]


%
\end{lstlisting}


\section{Using the \lstinline{fvlib} tools to create filevector files}
For those of you who would rather use 'regular' command line tools to
create filevector files the filevector library can be of help.

\subsection{Downloading and compiling the tools}
At the time of writing the latest release of the filevector library is
v1.0.1, which can be downloaded from Github and then extracted:
\begin{lstlisting}[basicstyle=\footnotesize\ttfamily]
wget https://github.com/GenABEL-Project/filevector/archive/v1.0.1.tar.gz
tar -xzf
\end{lstlisting}

The following steps compile the tools:
\begin{lstlisting}
cd filevector-1.0.1
mkdir bin
make
\end{lstlisting}
If compilation succeeds, the |bin/| directory should contain several
tools, of which |text2fvf| and |fv2text| are the most interesting. As
their names already hint, they are used to convert text files to
filevector format and the other way around, respectively.

\subsection{Converting text files to filevector format}
For our present purpose, creating input files for \oanomm, |text2fvf|
is the tool we need. Running the tool without any options lists some
usage information. The basic format is:
\begin{lstlisting}[basicstyle=\footnotesize\ttfamily]
bin/text2fvf -i INFILE -o OUTFILE [-r N2] [--colnames[=CFILE]] [-t] [-datatype=TYPE]
\end{lstlisting}
Of special note here is the |-datatype| or |-d| option, which defaults to
|DOUBLE| by default. For \oanomm |FLOAT| should be specified.

The following command converts a text file |pheno.txt| with phenotype
data in which the sample IDs are in the first column to filevector
format:
\begin{lstlisting}[basicstyle=\footnotesize\ttfamily]
bin/text2fvf -i pheno.txt -o pheno --rownames=1 -d FLOAT --skipcols=1
\end{lstlisting}
Note that the |--skipcols| option needs to be specified because the
first column contains the ID names (as specified by the |--rownames|
option).

%%% Local Variables:
%%% mode: latex
%%% TeX-master: "UserGuide"
%%% End:
