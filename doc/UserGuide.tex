\documentclass{report}

\usepackage{fullpage}

\usepackage{graphicx,color}
\usepackage{mhchem}
\usepackage{xcolor}
\usepackage{listings}


\lstdefinestyle{BASH}
{
    backgroundcolor=\color{black},
    basicstyle=\scriptsize\color{white}\ttfamily
}


\begin{document}

\title{OmicabelNoMM User's Guide}
\author{Alvaro Frank, NAME,NAME}
\date{October 2014}
\maketitle


%%%%%%%%%%%%%%%%%%%%%%%%%%%%%%%%%%%
\chapter{Quick Usage}


%%%%%%%%%%%%%%%%%%%%%%%%%%%%%%%%%%%
\chapter{Understanding OmicabelNoMM}

\section{Overview}

\section{Glossary}

\section{Formulas}

\subsection{Possible analysis}

\subsubsection{Basic analysis}
\begin{align}
y &\sim \beta_0 1 + \beta_1 x \\
y &\sim \beta_0 1 + \beta_1 cov_1 + \dots + \beta_l cov_l + \beta_r x_r\\
y &\sim \beta_0 1 + \beta_1 cov_1 + \dots + \beta_l cov_l + \beta_{l+1} x_{l+1}  + \dots + \beta_p x_p\\
y &\sim \beta_0 1 + \beta_1 cov_1 + \dots + \beta_l cov_l + \beta_{r} \left(x_{l+1}  + \dots +  x_p\right)
\end{align}

\subsubsection{Analysis with factors/dosages}
\begin{align}
y &\sim \beta_0 1 + \beta_1 cov_1 + \dots + \beta_l cov_l + \beta_r \phi_1 x_r\\
y &\sim \beta_0 1 + \beta_1 cov_1 + \dots + \beta_l cov_l + \beta_{l+1}  \phi_1 x_{l+1}  + \dots + \beta_p  \phi_r x_p\\
y &\sim \beta_0 1 + \beta_1 cov_1 + \dots + \beta_l cov_l + \beta_{r} \left( \phi_1 x_{l+1}  + \dots +   \phi_r x_p\right)
\end{align}

\subsubsection{Analysis with Interactions/Environmental Effects}
\begin{align}
y &\sim \beta_0 1 + \beta_1 cov_1 + \dots + \beta_l cov_l + \beta_r i_1 x_r\\
y &\sim \beta_0 1 + \beta_1 cov_1 + \dots + \beta_l cov_l + \beta_{l+1} i_1 x_r + \dots  + \beta_j j_1 x_r\\
y &\sim \beta_0 1 + \beta_1 cov_1 + \dots + \beta_l cov_l + \beta_{r}  i_1 \left( x_{l+1}  + \dots + x_p\right) \\
y &\sim \beta_0 1 + \beta_1 cov_1 + \dots + \beta_l cov_l + \beta_{l+1}  i_1 \left( x_{l+1}  + \dots + x_p\right) +\dots + \beta_{j}  i_j \left( x_{l+1}  + \dots + x_p\right) 
\end{align}

\subsubsection{Analysis with Interactions/Environmental Effects keeping original variable}
\begin{align}
y &\sim\beta_0 1 + \beta_1 cov_1 + \dots + \beta_l cov_l +  \beta_{l+1} x_r +  \beta_{l+2} i_1 x_r\\
y &\sim \beta_0 1 + \beta_1 cov_1 + \dots + \beta_l cov_l + \beta_{l+1} x_r + \beta_{l+2} i_1 x_r + \dots  + \beta_j j_1 x_r
\end{align}

\subsubsection{Analysis with Interactions and factors}
\begin{align}
y &\sim \beta_0 1 + \beta_1 cov_1 + \dots + \beta_l cov_l + \beta_r i_1 \phi_1 x_r\\
y &\sim \beta_0 1 + \beta_1 cov_1 + \dots + \beta_l cov_l + \beta_{l+1} i_1 \phi_1 x_r + \dots  + \beta_j j_1 \phi_1 x_r\\
y &\sim \beta_0 1 + \beta_1 cov_1 + \dots + \beta_l cov_l + \beta_{r}  i_1 \left( \phi_{l+1} x_{l+1}  + \dots + \phi_p x_p\right) \\
y &\sim  \dots + \beta_l cov_l + \beta_{l+1}  i_1 \left(  \phi_{l+1} x_{l+1}  + \dots +  \phi_{p} x_p\right) +\dots + \beta_{j}  i_j \left( \phi_{l+1} x_{l+1}  + \dots + \phi_{p} x_p\right) 
\end{align}

\subsubsection{Analysis with Interactions and factor keeping original variable}
\begin{align}
y &\sim\beta_0 1 + \beta_1 cov_1 + \dots + \beta_l cov_l +  \beta_{l+1} \phi_{r} x_r +  \beta_{l+2} i_1 \phi_{r} x_r\\
y &\sim \beta_0 1 + \beta_1 cov_1 + \dots + \beta_l cov_l + \beta_{l+1} \phi_{r} x_r + \beta_{l+2} i_1 \phi_{r} x_r + \dots  + \beta_j j_1 \phi_{r} x_r
\end{align}


\subsection{Regression Coefficients}

$\beta=(X^T X)^{-1} X^T y$

\subsection{T-statistic}
\subsection{P-values}


\section{Algorithm}

\section{Compromises}

%%%%%%%%%%%%%%%%%%%%%%%%%%%%%%%%%%%
\chapter{Setting OmicabelNoMM up}

\section{Setup a project}
\begin{lstlisting}[style=BASH,escapechar=\%]
#projects location
mkdir GWAS_PROJECT

cd GWAS_PROJECT
%
\end{lstlisting}

\section{Library and program Requirements}

\subsection{autoconf, autotools}

Make sure you have autoconf/autotools installed
\begin{lstlisting}[style=BASH,escapechar=\%]

sudo apt-get install autoconf
autoreconf -fi
autoconf
%
\end{lstlisting}

\subsection{Compilers}

You will need the latest gcc compiler for your system for running OmicABELnoMM on a single multi-core computer .

\begin{lstlisting}[style=BASH,escapechar=\%]

sudo apt-get install gcc-4.9
%
\end{lstlisting}

For compute-cluster you will need MPI support.

\begin{lstlisting}[style=BASH,escapechar=\%]
sudo apt-get install openmpi-bin
sudo apt-get install openmpi-common
sudo apt-get install libopenmpi
sudo apt-get install libopenmpi-dbg 
sudo apt-get install libopenmpi-dev
\end{lstlisting}

\subsection{Blas and Lapack}

You will need a Linear Algebra Library for high performance matrix computations.
The standard is to use openblas and lapack.

\begin{lstlisting}[style=BASH,escapechar=\%]

sudo apt-get install libopenblas-dev
sudo apt-get install libopenblas-base
sudo apt-get install liblapack3gf
sudo apt-get install liblapack-doc
sudo apt-get install liblapack-dev
sudo apt-get install liblapacke
sudo apt-get install liblapacke-dev
%
\end{lstlisting}

For alternative ways of installing BLAS and lapack, you can download the source code directly and compile for your own machine, guaranting that the settings will be optimal. Sometimes distributions lack USE\_OPENMP=1.
Remember to change path\_to\_ with your your own path to the specified folder.

\begin{lstlisting}[style=BASH,escapechar=\%]

git clone git://github.com/xianyi/OpenBLAS

cd OpenBLAS

#make sure you use g++ 4.8 or Higher!
make all HOSTCC=g++ FC=gfortran USE_OPENMP=1

#install the libraries relative to OmicABELnoMM
make install PREFIX="path_to_/OmicABELnoMM/libs/"
%
\end{lstlisting}
(Status: Support Broken)
You can Use AMD's ACML (BLAS from AMD) by going to:\\
http://developer.amd.com/tools-and-sdks/cpu-development/amd-core-math-library-acml/acml-downloads-resources/ \\
and copy the supplied binary libraries to "/OmicABELnoMM/libs/". IF both libraries are present (Openblas + ACML), the system will use ACML.

Let Omicabel know where BLAS is located by:


\begin{lstlisting}[style=BASH,escapechar=\%]

export LD_LIBRARY_PATH=$LD_LIBRARY_PATH:path_to_/OmicABELnoMM/libs/lib
autoreconf -fi

./configure
$%
\end{lstlisting}

\section{Source Files}

\begin{lstlisting}[style=BASH,escapechar=\%]

#get the source files
svn checkout svn://svn.r-forge.r-project.org/svnroot/genabel/pkg/OmicABELnoMM/

cd OmicABELnoMM
%
\end{lstlisting}

\section{Compiling}

For compiling the final executable binary use:
\begin{lstlisting}[style=BASH,escapechar=\%]

#in /OmicABELnoMM/
make
%
\end{lstlisting}

For compiling the test binary use:
\begin{lstlisting}[style=BASH,escapechar=\%]

#in /OmicABELnoMM/
make check
%
\end{lstlisting}

%%%%%%%%%%%%%%%%%%%%%%%%%%%%%%%%%%%
\chapter{Preparing Source Data}

\section{Overview}

OmicABELnoMM uses a DatABEL format for the source files. DatABEL uses less storage space, and helps computations to be done faster.

Original source files can be in any format as long as there is a way to load them into R for a table(matrix) format. Once in table format, they can be just transformed to DatABEL format to be used by OmicABEL.


\section{Databel}
Start R, then use library(DatABEL); help("DatABEL-package")\\
More info: http://www.genabel.org/packages/DatABEL\\
Start R and load DatABEL

\begin{lstlisting}[style=BASH,escapechar=\%]

library(DatABEL)
%
\end{lstlisting}

\section{Covariates}

This example shows how to artificially crate covariates:

\begin{lstlisting}[style=BASH,escapechar=\%]

#START_FAKE_DATA
n = 2000		 # number of individuals
l = 3   		 # number of covariates+1 for intercept
r = 2 			 # how many columns per SNP
m = r*100000		 # number of snps
t = 10000  		 # number of traits
set.seed(1001)
runif(3)
XL <- matrix(rnorm((l+1)*n),ncol=(l+1)) # first column should be ones (intercept)
for(i in 1:(n*(l+1))){ if(sample(1:100,1) > 95){XL[i]=0/0} }#fill in NANs
#END_FAKE_DATA

#FROM here on if you have your real data stored in the matrix variable XL you are ok.
#how to get your data into XL depends on your original files and how they were stored.

#The first column of covariates has to have 1's! it is the intercepts
#Make sure you add this column of ones and that you have the space for it
#without loosing your own data.
for(i in 1:n){ XL[i]=1}

#add your own idnames!
colnames(XL) <- c("intercept", paste("cov",1:l,sep=""))
rownames(XL) <- paste("ind",1:n,sep="")

#transform to databel (store it)
XL_db <- matrix2databel(XL,filename="XL",type="FLOAT")

#XL[1:n,1:(l+1)]
#XL
%
\end{lstlisting}

\section{Independent Variables, SNPs,CPG Sites,Measurements used to explain other Measurements}
\begin{lstlisting}[style=BASH,escapechar=\%]
#START_FAKE_DATA
n = 2000		 # number of individuals
l = 3   		 # number of covariates+1 for intercept
r = 2 			 # how many columns per SNP
m = r*100000		 # number of snps
t = 10000  		 # number of traits
#r=2
XR <- matrix(rnorm(m*n),ncol=m)

#Assumes that you had the previous Y still stored, this will create XR linearly dependent on the Y
for(i in 1 + r*(0:((m-2)/r)) )
{ 
	#print(i)
	yIdx=ceiling(i/r)
	#print(i)
	#print(yIdx)
	for(j in 1:n)
	{ 
		XR[j,i]=Y[j,yIdx]
		for(k in 1:l)
		{
			XR[j,i]=XR[j,i]-XL[j,k]*0.01
		}
		for(k in 1:(r-1))
		{
			XR[j,i]=XR[j,i]-XR[j,i+k]*0.01
		}
		#XR[j,i]=XR[j,i]/2.8888
		#XR[j,i] = XR[j,i]*runif(1, 1.0-var, 1.0)
		
	}
}

#add missing data
for(i in 1:(n*m)){ if(sample(1:100,1) > 90) XR[i]=0/0}
#END_FAKE_DATA

#FROM here on if you have your real data stored in the matrix variable XL you are ok.
#how to get your data into XL depends on your original files and how they were stored.

#The first column of covariates has to have 1's! it is the intercepts
#Make sure you add this column of ones and that you have the space for it
#without loosing your own data.

#add your own idnames!
colnames(XR) <- paste("miss",1:m,sep="")
for(i in 1:(m/r))
{
	for(j in 1:r) 
	{
		colnames(XR)[(i-1)*r+(j)] = paste0("snp",paste(i,j,sep="_") )
	}
}

#add your own idnames!
rownames(XR) <- paste("ind",1:n,sep="")

#transform to databel (store it)
XR_db <- matrix2databel(XR,filename="XR",type="FLOAT")
%
\end{lstlisting}

\section{Dependent Variable, Phenotypes,Measurements to be explained}

\begin{lstlisting}[style=BASH,escapechar=\%]


%
\end{lstlisting}

%%%%%%%%%%%%%%%%%%%%%%%%%%%%%%%%%%%
\chapter{Running Analysis}

\section{Getting help from the program}

\begin{lstlisting}[style=BASH,escapechar=\%]

./omicabelnomm -h
usage: omicabelnomm -c <path/fname> --geno <path/fname> -p <path/fname> -o <path/fname> 
                -x <path/fname> -n <#SNPcols> -t <#CPUs>
                         -d <0.0~1.0> -r <-10.0~1.0> -b -s <0.0~1.0>  -e <-10.0~1.0> -i -f
omicabelnomm Version 0.96b 
	Required: 
	-p --phe    	 <path/filename> to the inputs containing phenotypes. 
	-g --geno   	 <path/filename> to the inputs containing genotypes. 
	-c --cov    	 <path/filename> to the inputs containing covariates. 
	-o --out    	 <path/filename> to store the output to (used for all .txt and .ibin & .dbin). 

Optional: 
	-n --ngpred 	 <#SNPcols> Number of columns in the geno file that represent a single SNP. 
	-t --thr    	 <#CPUs> Number of computing threads to use to speed computations.
			 Recommended is 4-8 per node (see MPI). 
	-x --excl   	 <path/filename> file containing list of individuals to exclude 
			 from input files, (see example file). 
	-d --pdisp  	 <0.0~1.0> Value to use as maximum threshold for significance.
			 Results with P-values UNDER this threshold will be 
			 displayed in the putput .txt file. 
	-r --rdisp  	 <-10.0~1.0> Value to use as minimum threshold for R2. 
			 Results with R2-values ABOVE this threshold will be displayed
			 in the putput .txt file. 
	-b --stobin 	 Flag that forces to ALSO store results in a
			 smaller binary format (*.ibin & *.dbin). 
	-s --psto   	 <0.0~1.0>  Results with P-values UNDER this threshold will be 
			 displayed in the putput binary files. 
	-e --rsto   	 <-10.0~1.0> Results with R2-values ABOVE this threshold will be 
			 stored in the putput binary files. 
	-i --fdcov  	 Flag that forces to include covariates (when its genotype is significant) 
			 as part of the results stored 
	-f --fdgen  	 Flag that forces to consider all included results 
			 (causes the analisis to ignores ALL threshold values). 
	-j --additive  	 Flag that runs the analisis with an Additive Model with 
			 (2*AA,1*AB,0*BB) effects. 
	-k --dominant  	 Flag that runs the analisis with an Dominant Model with 
			 (1*AA,1*AB,0*BB) effects. 
	-l --recessive 	 Flag that runs the analisis with an Recessive Model with 
			 (1*AA,0*AB,0*BB) effects. 
	-z --mylinear 	 <path/filename> to read Factors 'f_i' for a Custom Linear Model with
			 f1*X1,f2*X2,f3*X3...fn*X_ngpred as effects,
			 each column of each independent variable will be multiplied with
			 the specified factors. 
			 Formula: y~alpha*cov + beta_1*f1*X1 + beta_2*f2*X2 +...+ beta_n*fn*Xn, 
			 (see example files!). 
%
\end{lstlisting}
\pagebreak
\begin{lstlisting}[style=BASH,escapechar=\%]

	-y --myaddit  	 <path/filename> to read Factors 'f_i' for a Custom Additive Model with
			 (f1*X1,f2*X2,f3*X3...fn*X_ngpred) as effects,
			 each column of each independent variable will be multiplied with the 
			 specified factors and then added together. 
			 Formula: y~alpha*cov + beta*(f1*X1 + f2*X2 +...+ fn*Xn), (see example files!).
	-v --simpleinter <path/filename> to read the interactions from;
			 for single analysis using multile interactions. 
	-w --multinter 	 <path/filename> to read the interactions from;
			 for multiple analysis using single interaction per analysis. 
	-u --keepinter 	 Flag that sets if the interaction analysis chose is to too keep the dependent 
			 variable X. If set, Formula: y~alpha*cov + beta_1*INT*X + beta_2*X, 
			 (see example files!).  Default not set, 
			 Formula: y~alpha*cov + beta_1*INT*X, (see example files!). 

			 Support for MPI is available. 
			 Simply use mpirun -np <#nodes> omicabelnomm <params> 
			 on an Open-MPI enabled computer/cluster.
			 Recommended is to use MPI when dealing with problems with over 2000 genotypes,
			 at a rate of 1 node per 2000 genotypes.
	

%
\end{lstlisting}

\section{WARNING: Theoretical Caveats}

\section{Simple Linear Regression}

Simple linear regression analysis with 4 threads can be done using (note long and short versions).
This setup assumes as default 1 column per XR (-n 1). In the default case, each column (-n 1) gets its own regression coefficient.
\begin{lstlisting}[style=BASH,escapechar=\%]

./omicabelnomm --cov examples/XL --geno examples/XR --phe examples/Y --out examples/B --thr 4

./omicabelnomm -c examples/XL -g examples/XR -p examples/Y -o examples/B -t 4
%
\end{lstlisting}

When using more than one column per snp, you specify the value with -n 3, where each column of XR gets its own regression coefficient, i.e: 

\begin{lstlisting}[style=BASH,escapechar=\%]

./omicabelnomm -c examples/XL -g examples/XR -p examples/Y -o examples/B -t 4 -n 3
%
\end{lstlisting}

For analysis involving snp's and dosage models, the following popular options are allowed:

\begin{lstlisting}[style=BASH,escapechar=\%]

./omicabelnomm -c examples/XL -g examples/XR -p examples/Y -o examples/B -t 4 --additive

./omicabelnomm -c examples/XL -g examples/XR -p examples/Y -o examples/B -t 4 --recessive

./omicabelnomm -c examples/XL -g examples/XR -p examples/Y -o examples/B -t 4 --dominant
%
\end{lstlisting}

\section{Custom Dosage Analysis}

When using custom dosages, you need to specify how many columns per snp are you using. You also have to specify the file from which the dosage factors will be read. The file has to contain 1 factor per column of the snp. 
Using --myaddit will cause for all columns to be multiplied by the specific factors and then added together. The resulting vector (1 per -n  of the snp) will obtain a collective regression coefficient.\\
Using --mylinear each single -n will obtain its own regression coefficient after being multiplied by the respective dosage factor.

\begin{lstlisting}[style=BASH,escapechar=\%]

./omicabelnomm -c examples/XL -g examples/XR -p examples/Y -o examples/B -t 4 
						-n 2 --myaddit examples/dosages_2.txt

./omicabelnomm -c examples/XL -g examples/XR -p examples/Y -o examples/B -t 4
						-n 1 --mylinear examples/dosages_1.txt
%
\end{lstlisting}


\section{MPI and Cluster usage for Simple Linear Regression}

Compute clusters offer multiple compute nodes(computers) where each has multi threading capabilities. On OmicABELnoMM compiled using MPI support, you could use mpirun to use multiple nodes at once. 10 nodes using 8 threads each:

\begin{lstlisting}[style=BASH,escapechar=\%]

mpirun -np 10 ./omicabelnomm -c examples/XL --g examples/XR -p examples/Y -o examples/B -t 8
%
\end{lstlisting}

In this case each process (1 per node specified using -np for a total of 10 in the example) will create a different outputfile named from B\_mpi1\_dis.txt ... B\_mpi10\_dis.txt


\section{Simple interactions of non linear terms, Enviromental Effects}



%%%%%%%%%%%%%%%%%%%%%%%%%%%%%%%%%%%
\chapter{FAQ}

\end{document}
