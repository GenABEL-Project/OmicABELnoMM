\documentclass[a4paper,toc=bibliography]{scrreprt}

\usepackage[utf8]{inputenc}
\usepackage[T1]{fontenc}
\usepackage{textcomp}
\usepackage[osf,sc]{mathpazo}
 \linespread{1.05} % Palatino needs more leading (space between lines)
\usepackage{microtype}
% Use Palatino for the section headings as well
\addtokomafont{sectioning}{\normalfont\bfseries}

\usepackage[smaller]{acronym}
%\usepackage{fullpage}

\usepackage{graphicx,color}
\usepackage{amsmath}

\usepackage[svgnames]{xcolor}
\definecolor{webgreen}{rgb}{0,.5,0}

\usepackage{listings}
\definecolor{lstbgcolor}{rgb}{0.9,0.9,0.9}
\lstset{
  tabsize=4,
  rulecolor=,
  basicstyle=\ttfamily,
  upquote=true,
  columns=fixed,
  showstringspaces=false,
  extendedchars=true,
  breaklines=true,
  breakatwhitespace,
  prebreak = \raisebox{0ex}[0ex][0ex]{\ensuremath{\hookleftarrow}},
  frame=single,
  showtabs=false,
  showspaces=false,
  showstringspaces=false,
  keywordstyle=\color[rgb]{0,0,1},
  commentstyle=\color[rgb]{0,0.4,0},
  stringstyle=\color[rgb]{0.5,0,1},
%  basicstyle=\footnotesize\ttfamily,
  backgroundcolor=\color{lstbgcolor},
}
\lstloadlanguages{bash,awk}
\lstMakeShortInline{|}


\usepackage[pdftex,hyperfootnotes=false,pdfpagelabels]{hyperref}
\hypersetup{%
  linktocpage=false, % If true the page numbers in the toc are links
  % instead of the section headings.
  pdfstartview=FitH,% pdfstartpage=3,
  breaklinks=true, pageanchor=true, %
  pdfpagemode=UseOutlines, plainpages=false, bookmarksnumbered, %
  bookmarksopen=true, bookmarksopenlevel=1, hypertexnames=true, %
  pdfhighlight=/O, %hyperfootnotes=true,%nesting=true,%frenchlinks,%
  pdfauthor={\textcopyright\ L.C.~Karssen, A.~Frank},
  pdfsubject={OmicABELnoMM User Guide},
  colorlinks=true, urlcolor=blue, linkcolor=blue, citecolor=webgreen %
}
% get the links to the figures and tables right:
\usepackage[all]{hypcap} % to be loaded after hyperref package

% lowercase letters as footnote numerals (to avoid confusion with powers).
\renewcommand{\thefootnote}{\alph{footnote}}


% Some newly defined commands and operators:
\DeclareMathOperator{\var}{\mathbf{var}}
\DeclareMathOperator{\cov}{\mathbf{cov}}
\newcommand{\eg}{e.g.~}
\newcommand{\ie}{i.e.~}
\usepackage{xspace}
\newcommand{\oanomm}{OmicABELnoMM\xspace}

\begin{document}

\title{OmicabelNoMM v0.1.0 User Guide}
\author{Lennart C. Karssen$^1$, Alvaro Frank$^{2,3}$\\
  {\small \textsuperscript{1} PolyOmica, Groningen, NL} \\
  {\small \textsuperscript{2} HelmholtzZentrum, München, DE}\\
  {\small \textsuperscript{3} AICES, RWTH, Aachen, DE}\\
}
\date{2016}
\maketitle
\tableofcontents


\chapter{Introduction}
\oanomm is a tool for running \acp{GWAS} on large omics data sets in a
computationally efficient manner. It is specifically designed to run
linear regression on thousands of phenotypes (\eg metabolomics data)
for thousands of individuals and millions of (imputed) genotypes
within a reasonable time frame (think days instead of months or
years).

%%% Local Variables:
%%% mode: latex
%%% TeX-master: "UserGuide"
%%% End:

\chapter{Quick Usage}

%%% Local Variables:
%%% mode: latex
%%% TeX-master: "UserGuide"
%%% End:

\chapter{Installing \oanomm}

There are two ways (or maybe three) to install \oanomm:
\begin{itemize}
\item Download a pre-compiled binary version of the tool.
\item Compile \oanomm from source
\item Compile \oanomm from source, including the libraries it depends
  on.
\end{itemize}
Option number one is the easiest, however it also means you are
running a version of \oanomm that is significantly slower then
possible. Because \oanomm was developed with \ac{HPC} in mind, it
works most efficient if it can use the specific features of the
\ac{CPU} of your machine(s), which means compiling it yourself.

For a fully optimised version of \oanomm the required libraries need
to be compiled specifically for your hardware as well. In a scientific
computing environment it may be that your system administrator has
already done this for you.

Details of the first option can be found in \S~\ref{sec:binaryinstall}
and details of options two and three are documented in
\S~\ref{sec:compile}.

\section{Installing the pre-compiled binary}
\label{sec:binaryinstall}


\section{Compiling \oanomm from source}
\label{sec:compile}
In order to install \oanomm the user should compile the source code of
\oanomm to create executable programs. In order to do so successfully,
several libraries need to be installed on the system.  These libraries
can be pre-installed on the system (e.g. by the system administrator)
or downloaded separately by the user. If one wants a fully optimised
version of \oanomm, those libraries should be compiled for the
specific computer architecture of the user as well, but pre-built
packages (\eg those provided by the Linux distribution) will work as
well.

In short the process of compiling \oanomm (after the requirements have
been installed) looks like this:
\begin{itemize}
\item Check for the presence of all requirements: |./configure|
\item Compile \oanomm: |make|
\item Install \oanomm: |make install|
\end{itemize}

The following tools and libraries are needed when compiling \oanomm
yourself. Most of the libraries contain highly optimised code for
linear algebra and other math.
\begin{itemize}
\item Compilers: A C++ and a FORTRAN compiler are needed (we tested
  GCC (v4.8 or higher) or CLANG, gfortran)
\item Libraries: \oanomm depends on the following libraries:
  \begin{itemize}
  \item \acs{BLAS} (required): a linear algebra library. We tested OpenBLAS,
  \item LAPACKe (required): The C interface to LAPACK, the linear
    algebra package.
  \item Boost (required): For the calculation of p-values \oanomm needs the
    Boost-math part of the Boost library\footnote{The Boost library
      can be foundt at \url{}}.
  \item \ac{ACML} (optional): if the \ac{ACML} is found,
    it will be used where possible.
  \item Intel \ac{MKL} (optional): if the \ac{MKL} is
    found, it will be used where possible.
  \item \acs{MPI} (optional): If the |./configure| step detects an
    \acf{MPI} library like OpenMPI or MPICH2, \oanomm will be compiled
    with the option to distribute the computations across multiple
    machines on a cluster.
  \end{itemize}
\end{itemize}


\subsection{Required libraries: \ac{BLAS} and LAPACKe}
In order to use \oanomm you will need a linear algebra library for
high performance matrix computations. The standard is to use OpenBLAS
and LAPACKe. The following sections describe three ways of installing
these libraries on your system, each with their own advantages and
disadvantages.

\subsubsection{Using the packages provided by your Linux distribution}
The easiest way to install the \ac{BLAS} and LAPACKe libraries is to
use the pre-compiled packages provided by your Linux distribution. For
example, on an Ubuntu Linux system these can be installed with the
following commands (other distributions will have similarly named
packages):
\begin{lstlisting}[escapechar=\%]
sudo apt-get install libopenblas-dev
sudo apt-get install libopenblas-base
sudo apt-get install liblapack3gf
sudo apt-get install liblapack-doc
sudo apt-get install liblapack-dev
sudo apt-get install liblapacke
sudo apt-get install liblapacke-dev
\end{lstlisting}
The downside of using this method is that the BLAS packages won't be
optimised for your system, probably leading to worse performance.


\subsubsection{Compiling and installing the libraries yourself}
An alternative way of installing \ac{BLAS} and LAPACKe, is to download
the source code directly and compile for your own machine,
guaranteeing that the settings will be optimally adjusted for your
hardware. One reason for compiling the \ac{BLAS} library yourself
instead of using the packages supplied by your Linux distribution is
that sometimes the distribution packages didn't use the |USE_OPENMP=1|
flag. Remember to change |path_to_| with your your own path to the
specified folder.

The latest release from the OpenBLAS library can be found at
\url{https://github.com/xianyi/OpenBLAS/releases}. At the time of
writing this is is version 0.2.15. Download the |tar.gz| file to the
server and decompress the archive:
\begin{lstlisting}[basicstyle=\footnotesize\ttfamily]
wget https://github.com/xianyi/OpenBLAS/archive/v0.2.15.tar.gz
tar -xzf v0.2.15.tar.gz
cd OpenBLAS-0.2.15
\end{lstlisting}

Alternatively, you can download the development version from GitHub:
\begin{lstlisting}[escapechar=\%]
git clone https://github.com/xianyi/OpenBLAS.git
cd OpenBLAS
\end{lstlisting}

Now the OpenBLAS library can be compiled\footnote{Make sure you use
  g++ 4.8 or higher!}:
\begin{lstlisting}[escapechar=\%]
make all USE_OPENMP=1
\end{lstlisting}
This will take a while. If you have multiple CPU cores at your
disposal, you can add the |-j4| flag to the make command to tell it to
use 4 cores (adjust as appropriate).

The next step is to install the library in a directory relative to
OmicABELnoMM's source code:
\begin{lstlisting}
make install PREFIX="path_to_/OmicABELnoMM/lib/"
\end{lstlisting}
Note that |path_to_| is to be defined by you, depending on where you
downloaded and extracted the \oanomm source code\footnote{Note that if
you are installing this as system administrator and want to allow all
users to use this library, using a central directory like
\lstinline{/usr/local/lib/} is more appropriate.}.

\subsubsection{Using AMD's \ac{ACML}}
\textbf{(Status: Support Broken)}

Instead of using OpenBLAS one can use AMD's \ac{ACML} (a highly optimised
\ac{BLAS} library from AMD) by going to:
\url{http://developer.amd.com/tools-and-sdks/cpu-development/amd-core-math-library-acml/acml-downloads-resources/}
and copy the supplied binary libraries to |/OmicABELnoMM/lib/|. If
both libraries are present (OpenBLAS + \ac{ACML}), the system will use
\ac{ACML}.


\subsection{Compiling \oanomm}
\label{sec:compiling-oanomm}
Now that the prerequisites have been installed \oanomm itself can be compiled.

First, let \oanomm know where the \ac{BLAS} library is located by
setting the |LD_LIBRARY_PATH| environment variable\footnote{If the
  \ac{BLAS} library is installed in a central 'default' directory
  (like \lstinline{/usr/local/lib/}), this step is not necessary}:
\begin{lstlisting}[basicstyle=\footnotesize\ttfamily]
export LD_LIBRARY_PATH=$LD_LIBRARY_PATH:path_to_/OmicABELnoMM/libs/lib
\end{lstlisting}
Again, make sure to replace |path_to_| with the proper path on your system.

The actual compilation and installation of \oanomm is done by running
the following three commands, which will be explained in more detail
below:
\begin{itemize}
\item |./configure|
\item |make|
\item |make install|
\end{itemize}
The |./configure| step checks whether all required software like
libraries are in place. If not it will warn or even abort. If one or
more of the required libraries can not be found in a standard location
the correct location can be added as an option to the configure step,
for example to point to a different location of the Boost math library
run:
\begin{lstlisting}
./configure --with-boost-include-path=/path/to/boost/lib
\end{lstlisting}
The names of the options for each library can be found by typing
\begin{lstlisting}
./configure --help
\end{lstlisting}

Apart from the library locations, you can specify the installation
location in the configure step. By default, \oanomm will be installed
in the |/usr/local/| directory\footnote{In more detail: the binaries
  will be installed in \lstinline{/usr/local/bin/}, documentation in
  \lstinline{/usr/local/share/omicabelnomm/doc}.}. To change this,
use the |--prefix| option. For example, to install in a subdirectory
of your home directory run
\begin{lstlisting}
./configure --prefix=/home/yourusername/mytools/OmicABELnoMM
\end{lstlisting}


\section{Installing the development version}
For those of you who want to aid in the development of \oanomm

\subsection{autoconf, autotools}

Make sure you have autoconf/autotools installed
\begin{lstlisting}[escapechar=\%]

sudo apt-get install autoconf
autoreconf -fi
autoconf
%
\end{lstlisting}


\section{Setup a project}
\begin{lstlisting}[escapechar=\%]
#projects location
mkdir GWAS_PROJECT

cd GWAS_PROJECT
%
\end{lstlisting}

\section{Library and program Requirements}


\subsection{Compilers}

You will need the latest gcc compiler for your system for running \oanomm on a single multi-core computer .

\begin{lstlisting}[escapechar=\%]
sudo apt-get install gcc-4.9
%
\end{lstlisting}

For compute-cluster you will need \ac{MPI} support.

\begin{lstlisting}[escapechar=\%]
sudo apt-get install openmpi-bin
sudo apt-get install openmpi-common
sudo apt-get install libopenmpi
sudo apt-get install libopenmpi-dbg
sudo apt-get install libopenmpi-dev
\end{lstlisting}


\section{Source Files}

\begin{lstlisting}[basicstyle=\footnotesize\ttfamily,]
svn checkout svn://svn.r-forge.r-project.org/svnroot/genabel/pkg/OmicABELnoMM/

cd OmicABELnoMM
%
\end{lstlisting}

\section{Compiling}

For compiling the final executable binary use:
\begin{lstlisting}[escapechar=\%]
#in /OmicABELnoMM/
make
%
\end{lstlisting}

For compiling the test binary use:
\begin{lstlisting}[escapechar=\%]

#in /OmicABELnoMM/
make check
%
\end{lstlisting}

%%% Local Variables:
%%% mode: latex
%%% TeX-master: "UserGuide"
%%% End:

\chapter{Preparing input data}

\section{Overview}

\oanomm uses the filevector (also called DatABEL) file format for the
input files. Filevector files use less storage space than plain text
files and stores data in a way that make the columns and rows easily
and quickly addressable, and thus helps the computations to be done
faster\footnote{The source code of the filevector libary
  \lstinline{fvlib} can be found on Github:
  \url{https://github.com/GenABEL-Project/filevector}. This repository
  also contains a few command line utilities for manipulate filevector
  files (see \S)}. DatABEL is an R package providing an easy way to work
with filevector files\footnote{The DatABEL page on the GenABEL project
  website is \url{http://www.genabel.org/packages/DatABEL}. The R
  package can be downloaded from CRAN:
  \url{https://cran.r-project.org/web/packages/DatABEL/}.}.


Original source files can be in any format as long as there is a way
to load them into R for a table(matrix) format. Once in table format,
they can be just transformed to DatABEL format to be used by OmicABEL.


\section{Using DatABEL to convert input files to filevector format}
Start R, then use

library(DatABEL); help("DatABEL-package")\\
More info: http://www.genabel.org/packages/DatABEL\\
Start R and load DatABEL

\begin{lstlisting}[escapechar=\%]
library(DatABEL)
\end{lstlisting}

\subsection{Covariates}

The following example code shows how to artificially create covariates:
\begin{lstlisting}[escapechar=\%]
#START_FAKE_DATA
n = 2000                 # number of individuals
l = 3                    # number of covariates+1 for intercept
r = 2                    # how many columns per SNP
m = r*100000             # number of snps
t = 10000                # number of traits
set.seed(1001)
runif(3)
XL <- matrix(rnorm((l+1)*n), ncol=(l+1)) # first column should be ones (intercept)
for (i in 1:(n * (l+1))) {
    if (sample(1:100, 1) > 95){
        XL[i]=0/0    # fill in NANs
    }
}
#END_FAKE_DATA

# From here on if you have your real data stored in the matrix
# variable XL you are ok.
# How to get your data into XL depends on your original files and
# how they were stored.

# The first column of covariates has to have 1's! it is the intercept
# Make sure you add this column of ones and that you have the space for it
# without loosing your own data.
for (i in 1:n) {
   XL[i] = 1
}

# Add your own idnames!
colnames(XL) <- c("intercept", paste("cov", 1:l, sep=""))
rownames(XL) <- paste("ind", 1:n, sep="")

# Convert to databel (i.e. store the data in a file)
XL_db <- matrix2databel(XL,filename="XL",type="FLOAT")

#XL[1:n,1:(l+1)]
#XL
%
\end{lstlisting}

\subsection{Independent Variables, SNPs,CPG Sites,Measurements used to explain other Measurements}
\begin{lstlisting}[escapechar=\%]
#START_FAKE_DATA
n = 2000                 # number of individuals
l = 3                    # number of covariates+1 for intercept
r = 2                    # how many columns per SNP
m = r*100000             # number of snps
t = 10000                # number of traits
#r=2
XR <- matrix(rnorm(m*n),ncol=m)

#Assumes that you had the previous Y still stored, this will create XR linearly dependent on the Y
for(i in 1 + r*(0:((m-2)/r)) )
{
        #print(i)
        yIdx=ceiling(i/r)
        #print(i)
        #print(yIdx)
        for(j in 1:n)
        {
                XR[j,i]=Y[j,yIdx]
                for(k in 1:l)
                {
                        XR[j,i]=XR[j,i]-XL[j,k]*0.01
                }
                for(k in 1:(r-1))
                {
                        XR[j,i]=XR[j,i]-XR[j,i+k]*0.01
                }
                #XR[j,i]=XR[j,i]/2.8888
                #XR[j,i] = XR[j,i]*runif(1, 1.0-var, 1.0)

        }
}

#add missing data
for(i in 1:(n*m)){ if(sample(1:100,1) > 90) XR[i]=0/0}
#END_FAKE_DATA

#FROM here on if you have your real data stored in the matrix variable XL you are ok.
#how to get your data into XL depends on your original files and how they were stored.

#The first column of covariates has to have 1's! it is the intercepts
#Make sure you add this column of ones and that you have the space for it
#without loosing your own data.

#add your own idnames!
colnames(XR) <- paste("miss",1:m,sep="")
for(i in 1:(m/r))
{
        for(j in 1:r)
        {
                colnames(XR)[(i-1)*r+(j)] = paste0("snp",paste(i,j,sep="_") )
        }
}

#add your own idnames!
rownames(XR) <- paste("ind",1:n,sep="")

#transform to databel (store it)
XR_db <- matrix2databel(XR,filename="XR",type="FLOAT")
%
\end{lstlisting}

\subsection{Dependent Variable, Phenotypes,Measurements to be explained}

\begin{lstlisting}[escapechar=\%]


%
\end{lstlisting}


\section{Using the \lstinline{fvlib} tools to create filevector files}
For those of you who would rather use 'regular' command line tools to
create filevector files the filevector library can be of help.

\subsection{Downloading and compiling the tools}
At the time of writing the latest release of the filevector library is
v1.0.1, which can be downloaded from Github and then extracted:
\begin{lstlisting}[basicstyle=\footnotesize\ttfamily]
wget https://github.com/GenABEL-Project/filevector/archive/v1.0.1.tar.gz
tar -xzf
\end{lstlisting}

The following steps compile the tools:
\begin{lstlisting}
cd filevector-1.0.1
mkdir bin
make
\end{lstlisting}
If compilation succeeds, the |bin/| directory should contain several
tools, of which |text2fvf| and |fv2text| are the most interesting. As
their names already hint, they are used to convert text files to
filevector format and the other way around, respectively.

\subsection{Converting text files to filevector format}
For our present purpose, creating input files for \oanomm, |text2fvf|
is the tool we need. Running the tool without any options lists some
usage information. The basic format is:
\begin{lstlisting}[basicstyle=\footnotesize\ttfamily]
bin/text2fvf -i INFILE -o OUTFILE [-r N2] [--colnames[=CFILE]] [-t] [-datatype=TYPE]
\end{lstlisting}
Of special note here is the |-datatype| or |-d| option, which defaults to
|DOUBLE| by default. For \oanomm |FLOAT| should be specified.

The following command converts a text file |pheno.txt| with phenotype
data in which the sample IDs are in the first column to filevector
format:
\begin{lstlisting}[basicstyle=\footnotesize\ttfamily]
bin/text2fvf -i pheno.txt -o pheno --rownames=1 -d FLOAT --skipcols=1
\end{lstlisting}
Note that the |--skipcols| option needs to be specified because the
first column contains the ID names (as specified by the |--rownames|
option).

%%% Local Variables:
%%% mode: latex
%%% TeX-master: "UserGuide"
%%% End:

\chapter{Running Analysis}

\section{Getting help from the program}

\begin{lstlisting}[basicstyle=\footnotesize\ttfamily]

./omicabelnomm -h
usage: omicabelnomm -c <path/fname> --geno <path/fname> -p <path/fname> -o <path/fname>
                -x <path/fname> -n <#SNPcols> -t <#CPUs>
                         -d <0.0~1.0> -r <-10.0~1.0> -b -s <0.0~1.0>  -e <-10.0~1.0> -i -f
omicabelnomm Version 0.96b
        Required:
        -p --phe         <path/filename> to the inputs containing phenotypes.
        -g --geno        <path/filename> to the inputs containing genotypes.
        -c --cov         <path/filename> to the inputs containing covariates.
        -o --out         <path/filename> to store the output to (used for all .txt and .ibin & .dbin).

Optional:
        -n --ngpred      <#SNPcols> Number of columns in the geno file that represent a single SNP.
        -t --thr         <#CPUs> Number of computing threads to use to speed computations.
                         Recommended is 4-8 per node (see MPI).
        -x --excl        <path/filename> file containing list of individuals to exclude
                         from input files, (see example file).
        -d --pdisp       <0.0~1.0> Value to use as maximum threshold for significance.
                         Results with P-values UNDER this threshold will be
                         displayed in the putput .txt file.
        -r --rdisp       <-10.0~1.0> Value to use as minimum threshold for R2.
                         Results with R2-values ABOVE this threshold will be displayed
                         in the putput .txt file.
        -b --stobin      Flag that forces to ALSO store results in a
                         smaller binary format (*.ibin & *.dbin).
        -s --psto        <0.0~1.0>  Results with P-values UNDER this threshold will be
                         displayed in the putput binary files.
        -e --rsto        <-10.0~1.0> Results with R2-values ABOVE this threshold will be
                         stored in the putput binary files.
        -i --fdcov       Flag that forces to include covariates (when its genotype is significant)
                         as part of the results stored
        -f --fdgen       Flag that forces to consider all included results
                         (causes the analisis to ignores ALL threshold values).
        -j --additive    Flag that runs the analisis with an Additive Model with
                         (2*AA,1*AB,0*BB) effects.
        -k --dominant    Flag that runs the analisis with an Dominant Model with
                         (1*AA,1*AB,0*BB) effects.
        -l --recessive   Flag that runs the analisis with an Recessive Model with
                         (1*AA,0*AB,0*BB) effects.
        -z --mylinear    <path/filename> to read Factors 'f_i' for a Custom Linear Model with
                         f1*X1,f2*X2,f3*X3...fn*X_ngpred as effects,
                         each column of each independent variable will be multiplied with
                         the specified factors.
                         Formula: y~alpha*cov + beta_1*f1*X1 + beta_2*f2*X2 +...+ beta_n*fn*Xn,
                         (see example files!).
%
\end{lstlisting}
\pagebreak
\begin{lstlisting}[escapechar=\%]

        -y --myaddit     <path/filename> to read Factors 'f_i' for a Custom Additive Model with
                         (f1*X1,f2*X2,f3*X3...fn*X_ngpred) as effects,
                         each column of each independent variable will be multiplied with the
                         specified factors and then added together.
                         Formula: y~alpha*cov + beta*(f1*X1 + f2*X2 +...+ fn*Xn), (see example files!).
        -v --simpleinter <path/filename> to read the interactions from;
                         for single analysis using multile interactions.
        -w --multinter   <path/filename> to read the interactions from;
                         for multiple analysis using single interaction per analysis.
        -u --keepinter   Flag that sets if the interaction analysis chose is to too keep the dependent
                         variable X. If set, Formula: y~alpha*cov + beta_1*INT*X + beta_2*X,
                         (see example files!).  Default not set,
                         Formula: y~alpha*cov + beta_1*INT*X, (see example files!).

                         Support for MPI is available.
                         Simply use mpirun -np <#nodes> omicabelnomm <params>
                         on an Open-MPI enabled computer/cluster.
                         Recommended is to use MPI when dealing with problems with over 2000 genotypes,
                         at a rate of 1 node per 2000 genotypes.


%
\end{lstlisting}

\section{WARNING: Theoretical Caveats}

\section{Simple Linear Regression}

Simple linear regression analysis with 4 threads can be done using (note long and short versions).
This setup assumes as default 1 column per XR (-n 1). In the default case, each column (-n 1) gets its own regression coefficient.
\begin{lstlisting}[escapechar=\%]

./omicabelnomm --cov examples/XL --geno examples/XR --phe examples/Y --out examples/B --thr 4

./omicabelnomm -c examples/XL -g examples/XR -p examples/Y -o examples/B -t 4
%
\end{lstlisting}

When using more than one column per snp, you specify the value with -n 3, where each column of XR gets its own regression coefficient, i.e:

\begin{lstlisting}[escapechar=\%]

./omicabelnomm -c examples/XL -g examples/XR -p examples/Y -o examples/B -t 4 -n 3
%
\end{lstlisting}

For analysis involving snp's and dosage models, the following popular options are allowed:

\begin{lstlisting}[escapechar=\%]

./omicabelnomm -c examples/XL -g examples/XR -p examples/Y -o examples/B -t 4 --additive

./omicabelnomm -c examples/XL -g examples/XR -p examples/Y -o examples/B -t 4 --recessive

./omicabelnomm -c examples/XL -g examples/XR -p examples/Y -o examples/B -t 4 --dominant
%
\end{lstlisting}

\section{Custom Dosage Analysis}

When using custom dosages, you need to specify how many columns per snp are you using. You also have to specify the file from which the dosage factors will be read. The file has to contain 1 factor per column of the snp.
Using --myaddit will cause for all columns to be multiplied by the specific factors and then added together. The resulting vector (1 per -n  of the snp) will obtain a collective regression coefficient.\\
Using --mylinear each single -n will obtain its own regression coefficient after being multiplied by the respective dosage factor.

\begin{lstlisting}[escapechar=\%]

./omicabelnomm -c examples/XL -g examples/XR -p examples/Y -o examples/B -t 4
                                                -n 2 --myaddit examples/dosages_2.txt

./omicabelnomm -c examples/XL -g examples/XR -p examples/Y -o examples/B -t 4
                                                -n 1 --mylinear examples/dosages_1.txt
%
\end{lstlisting}


\section{\ac{MPI} and Cluster usage for Simple Linear Regression}

Compute clusters offer multiple compute nodes(computers) where each has multi threading capabilities. On \oanomm compiled using \ac{MPI} support, you could use mpirun to use multiple nodes at once. 10 nodes using 8 threads each:

\begin{lstlisting}[escapechar=\%]

mpirun -np 10 ./omicabelnomm -c examples/XL --g examples/XR -p examples/Y -o examples/B -t 8
%
\end{lstlisting}

In this case each process (1 per node specified using -np for a total of 10 in the example) will create a different outputfile named from B\_mpi1\_dis.txt ... B\_mpi10\_dis.txt


\section{Simple interactions of non linear terms, Enviromental Effects}

%%% Local Variables:
%%% mode: latex
%%% TeX-master: "UserGuide"
%%% End:

\chapter{Understanding \oanomm}

\section{Overview}

\section{Glossary}

\section{Formulas}

\subsection{Possible analysis}

\subsubsection{Basic analysis}
\begin{align}
y &\sim \beta_0 1 + \beta_1 x \\
y &\sim \beta_0 1 + \beta_1 cov_1 + \dots + \beta_l cov_l + \beta_r x_r\\
y &\sim \beta_0 1 + \beta_1 cov_1 + \dots + \beta_l cov_l + \beta_{l+1} x_{l+1}  + \dots + \beta_p x_p\\
y &\sim \beta_0 1 + \beta_1 cov_1 + \dots + \beta_l cov_l + \beta_{r} \left(x_{l+1}  + \dots +  x_p\right)
\end{align}

\subsubsection{Analysis with factors/dosages}
\begin{align}
y &\sim \beta_0 1 + \beta_1 cov_1 + \dots + \beta_l cov_l + \beta_r \phi_1 x_r\\
y &\sim \beta_0 1 + \beta_1 cov_1 + \dots + \beta_l cov_l + \beta_{l+1}  \phi_1 x_{l+1}  + \dots + \beta_p  \phi_r x_p\\
y &\sim \beta_0 1 + \beta_1 cov_1 + \dots + \beta_l cov_l + \beta_{r} \left( \phi_1 x_{l+1}  + \dots +   \phi_r x_p\right)
\end{align}

\subsubsection{Analysis with Interactions/Environmental Effects}
\begin{align}
y &\sim \beta_0 1 + \beta_1 cov_1 + \dots + \beta_l cov_l + \beta_r i_1 x_r\\
y &\sim \beta_0 1 + \beta_1 cov_1 + \dots + \beta_l cov_l + \beta_{l+1} i_1 x_r + \dots  + \beta_j j_1 x_r\\
y &\sim \beta_0 1 + \beta_1 cov_1 + \dots + \beta_l cov_l + \beta_{r}  i_1 \left( x_{l+1}  + \dots + x_p\right) \\
y &\sim \beta_0 1 + \beta_1 cov_1 + \dots + \beta_l cov_l + \beta_{l+1}  i_1 \left( x_{l+1}  + \dots + x_p\right) +\dots + \beta_{j}  i_j \left( x_{l+1}  + \dots + x_p\right)
\end{align}

\subsubsection{Analysis with Interactions/Environmental Effects keeping original variable}
\begin{align}
y &\sim\beta_0 1 + \beta_1 cov_1 + \dots + \beta_l cov_l +  \beta_{l+1} x_r +  \beta_{l+2} i_1 x_r\\
y &\sim \beta_0 1 + \beta_1 cov_1 + \dots + \beta_l cov_l + \beta_{l+1} x_r + \beta_{l+2} i_1 x_r + \dots  + \beta_j j_1 x_r
\end{align}

\subsubsection{Analysis with Interactions and factors}
\begin{align}
y &\sim \beta_0 1 + \beta_1 cov_1 + \dots + \beta_l cov_l + \beta_r i_1 \phi_1 x_r\\
y &\sim \beta_0 1 + \beta_1 cov_1 + \dots + \beta_l cov_l + \beta_{l+1} i_1 \phi_1 x_r + \dots  + \beta_j j_1 \phi_1 x_r\\
y &\sim \beta_0 1 + \beta_1 cov_1 + \dots + \beta_l cov_l + \beta_{r}  i_1 \left( \phi_{l+1} x_{l+1}  + \dots + \phi_p x_p\right) \\
y &\sim  \dots + \beta_l cov_l + \beta_{l+1}  i_1 \left(  \phi_{l+1} x_{l+1}  + \dots +  \phi_{p} x_p\right) +\dots + \beta_{j}  i_j \left( \phi_{l+1} x_{l+1}  + \dots + \phi_{p} x_p\right)
\end{align}

\subsubsection{Analysis with Interactions and factor keeping original variable}
\begin{align}
y &\sim\beta_0 1 + \beta_1 cov_1 + \dots + \beta_l cov_l +  \beta_{l+1} \phi_{r} x_r +  \beta_{l+2} i_1 \phi_{r} x_r\\
y &\sim \beta_0 1 + \beta_1 cov_1 + \dots + \beta_l cov_l + \beta_{l+1} \phi_{r} x_r + \beta_{l+2} i_1 \phi_{r} x_r + \dots  + \beta_j j_1 \phi_{r} x_r
\end{align}


\subsection{Regression Coefficients}

$\beta=(X^T X)^{-1} X^T y$

\subsection{T-statistic}
\subsection{P-values}


\section{Algorithm}

\section{Compromises}


%%% Local Variables:
%%% mode: latex
%%% TeX-master: "UserGuide"
%%% End:

\chapter{FAQ}

%%% Local Variables:
%%% mode: latex
%%% TeX-master: "UserGuide"
%%% End:

\chapter{Acknowledgements and citation}
We thank the following people for their help and support in creating
\oanomm: Prof.~Paolo Bientinesi (AICES, RWTH, Aachen), Prof.~Christian
Gieger (HMGU, München), Sodbo Sharapov (Novosibirsk State University), ...

Without the financial support from the following sources, this work
would have been impossible: European Union FP7 framework projects
MIMOmics (grant agreement nr.~305280) and Pain-Omics (grant agreement
nr.~602736).

If you use \oanomm, please cite the following reference: XXX

The algorithmic background of \oanomm is described in \cite{Frank2016}.


%%% Local Variables:
%%% mode: latex
%%% TeX-master: "UserGuide"
%%% End:


\chapter{List of acronyms}
\label{ch:acro}

\begin{acronym}[BLAS] % Put longest acronym here for alignment
  \acro{ACML}{AMD Core Math Library\acroextra{, a highly optimised math
      library developed by AMD}}
  \acro{BLAS}{Basic Linear Algebra Subprograms\acroextra{, a type of
      library for efficient matrix operations, for example OpenBLAS}}
  \acro{CPU}{Central Processing Unit\acroextra{, i.e. the processor of
    your computer}}
  \acro{GWAS}{Genome-Wide Association Scan}
  \acro{HPC}{High-Performance Computing}
  \acro{MKL}{Math Kernel Library\acroextra{, a highly optimised math
      library developed by Intel}}
  \acro{MPI}{Message Passing Interface\acroextra{, library for
      distributed memory parallelisation that allows a program to be
      run in parallel on multiple machines in a cluster}}
\end{acronym}

\begin{thebibliography}{99}
\bibitem{Frank2016}A. Frank, D. Fabregat-Traver, and P. Bientinesi,
  \emph{Large-scale linear regression: Development of high-performance
    routines}, Applied Mathematics and Computation, v275, 2016,
  \href{http://dx.doi.org/10.1016/j.amc.2015.11.078}{doi:10.1016/j.amc.2015.11.078}
\end{thebibliography}

\end{document}

%%% Local Variables:
%%% mode: latex
%%% TeX-master: t
%%% End:
