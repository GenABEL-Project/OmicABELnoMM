\chapter{Introduction}
\oanomm is a tool for running \acp{GWAS} on large omics data sets in a
computationally efficient manner. It is specifically designed to run
linear regression on thousands of phenotypes (\eg metabolomics data)
for thousands of individuals and millions of (imputed) genotypes
within a reasonable time frame (think days instead of months or
years).

\section{Support}
Community support for \oanomm can be found on the
\href{http://forum.genabel.org}{GenABEL forum}. Feel free to ask your
questions there and to contribute with your own knowledge and skills.

Commercial support for \oanomm can be obtained via
\href{http://www.polyomica.com}{PolyOmica}. Typical things PolyOmica
can help you with are installing \oanomm on your server or
cluster, or the development of custom features.

\section{Contributing}
\oanomm is a free software project, we therefore welcome contributions
of all kind. Development takes place
\href{https://github.com/GenABEL-Project/OmicABELnoMM}{on GitHub},
where you will find the source code and
\href{https://github.com/GenABEL-Project/OmicABELnoMM/issues}{the list
  of bugs}. Feel free to submit your bug reports or feature requests
there, or, even better, fork the project and start fixing those bugs.
Development discussions take place on
\href{https://lists.r-forge.r-project.org/mailman/listinfo/genabel-devel}{the
  GenABEL development mailing list}. We are looking forward to working
with you!

\section{Licence}
\oanomm is free software licensed under the
\href{https://www.gnu.org/licenses/gpl.html}{GNU Generap Public
  License v3}. A copy of the licence can be found in the file
|COPYING|.



%%% Local Variables:
%%% mode: latex
%%% TeX-master: "UserGuide"
%%% End:
