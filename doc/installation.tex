\chapter{Installing \oanomm}

There are two ways (or maybe three) to install \oanomm:
\begin{itemize}
\item Download a pre-compiled binary version of the tool.
\item Compile \oanomm from source
\item Compile \oanomm from source, including the libraries it depends
  on.
\end{itemize}
Option number one is the easiest, however it also means you are
running a version of \oanomm that is significantly slower then
possible. Because \oanomm was developed with \ac{HPC} in mind, it
works most efficient if it can use the specific features of the
\ac{CPU} of your machine(s), which means compiling it yourself.

For a fully optimised version of \oanomm the required libraries need
to be compiled specifically for your hardware as well. In a scientific
computing environment it may be that your system administrator has
already done this for you.

Details of the first option can be found in \S~\ref{sec:binaryinstall}
and details of options two and three are documented in
\S~\ref{sec:compile}.

\section{Installing the pre-compiled binary}
\label{sec:binaryinstall}


\section{Compiling \oanomm from source}
\label{sec:compile}
In order to install \oanomm the user should compile the source code of
\oanomm to create executable programs. In order to do so successfully,
several libraries need to be installed on the system.  These libraries
can be pre-installed on the system (e.g. by the system administrator)
or downloaded separately by the user. If one wants a fully optimised
version of \oanomm, those libraries should be compiled for the
specific computer architecture of the user as well, but pre-built
packages (\eg those provided by the Linux distribution) will work as
well.

In short the process of compiling \oanomm (after the requirements have
been installed) looks like this:
\begin{itemize}
\item Check for the presence of all requirements: |./configure|
\item Compile \oanomm: |make|
\item Install \oanomm: |make install|
\end{itemize}

The following tools and libraries are needed when compiling \oanomm
yourself. Most of the libraries contain highly optimised code for
linear algebra and other math.
\begin{itemize}
\item Compilers: A C++ and a FORTRAN compiler are needed (we tested
  GCC (v4.8 or higher) or CLANG, gfortran)
\item Libraries: \oanomm depends on the following libraries:
  \begin{itemize}
  \item \acs{BLAS} (required): a linear algebra library. We tested OpenBLAS,
  \item LAPACKe (required): The C interface to LAPACK, the linear
    algebra package.
  \item Boost (required): For the calculation of $p$-values \oanomm
    needs the Boost-math part of the Boost library\footnote{The Boost
      library can be foundt at \url{http://www.boost.org}}.
  \item \ac{ACML} (optional): if the \ac{ACML} is found,
    it will be used where possible.
  \item Intel \ac{MKL} (optional): if the \ac{MKL} is
    found, it will be used where possible.
  \item \acs{MPI} (optional): If the |./configure| step detects an
    \acf{MPI} library like OpenMPI or MPICH2, \oanomm will be compiled
    with the option to distribute the computations across multiple
    machines on a cluster.
  \end{itemize}
\end{itemize}


\subsection{Required libraries: \ac{BLAS} and LAPACKe}
In order to use \oanomm you will need a linear algebra library for
high performance matrix computations. The standard is to use OpenBLAS
and LAPACKe. The following sections describe three ways of installing
these libraries on your system, each with their own advantages and
disadvantages.

\subsubsection{Using the packages provided by your Linux distribution}
The easiest way to install the \ac{BLAS} and LAPACKe libraries is to
use the pre-compiled packages provided by your Linux distribution. For
example, on an Ubuntu Linux system these can be installed with the
following commands (other distributions will have similarly named
packages):
\begin{lstlisting}[escapechar=\%]
sudo apt-get install libopenblas-dev
sudo apt-get install libopenblas-base
sudo apt-get install liblapack3gf
sudo apt-get install liblapack-doc
sudo apt-get install liblapack-dev
sudo apt-get install liblapacke
sudo apt-get install liblapacke-dev
\end{lstlisting}
The downside of using this method is that the BLAS packages won't be
optimised for your system, probably leading to worse performance.


\subsubsection{Compiling and installing the libraries yourself}
An alternative way of installing \ac{BLAS} and LAPACKe, is to download
the source code directly and compile for your own machine,
guaranteeing that the settings will be optimally adjusted for your
hardware. One reason for compiling the \ac{BLAS} library yourself
instead of using the packages supplied by your Linux distribution is
that sometimes the distribution packages didn't use the |USE_OPENMP=1|
flag. Remember to change |path_to_| with your your own path to the
specified folder.

The latest release from the OpenBLAS library can be found at
\url{https://github.com/xianyi/OpenBLAS/releases}. At the time of
writing this is is version 0.2.18. Download the |tar.gz| file to the
server and decompress the archive:
\begin{lstlisting}[basicstyle=\footnotesize\ttfamily]
wget https://github.com/xianyi/OpenBLAS/archive/v0.2.18.tar.gz
tar -xzf v0.2.18.tar.gz
cd OpenBLAS-0.2.18
\end{lstlisting}

Alternatively, you can download the development version from GitHub:
\begin{lstlisting}[escapechar=\%]
git clone https://github.com/xianyi/OpenBLAS.git
cd OpenBLAS
\end{lstlisting}

Now the OpenBLAS library can be compiled\footnote{Make sure you use
  g++ 4.8 or higher!}:
\begin{lstlisting}[escapechar=\%]
make all USE_OPENMP=1
\end{lstlisting}
This will take a while. If you have multiple CPU cores at your
disposal, you can add the |-j4| flag to the make command to tell it to
use 4 cores (adjust as appropriate).

The next step is to install the library in a directory relative to
OmicABELnoMM's source code:
\begin{lstlisting}
make install PREFIX="path_to_/OmicABELnoMM/lib/"
\end{lstlisting}
Note that |path_to_| is to be defined by you, depending on where you
downloaded and extracted the \oanomm source code\footnote{Note that if
you are installing this as system administrator and want to allow all
users to use this library, using a central directory like
\lstinline{/usr/local/lib/} is more appropriate.}.

\subsubsection{Using AMD's \ac{ACML}}
\textbf{(Status: Support Broken)}

Instead of using OpenBLAS one can use AMD's \ac{ACML} (a highly optimised
\ac{BLAS} library from AMD) by going to:
\url{http://developer.amd.com/tools-and-sdks/cpu-development/amd-core-math-library-acml/acml-downloads-resources/}
and copy the supplied binary libraries to |/OmicABELnoMM/lib/|. If
both libraries are present (OpenBLAS + \ac{ACML}), the system will use
\ac{ACML}.


\subsection{Compiling \oanomm}
\label{sec:compiling-oanomm}
Now that the prerequisites have been installed \oanomm itself can be compiled.

First, let \oanomm know where the \ac{BLAS} library is located by
setting the |LD_LIBRARY_PATH| environment variable\footnote{If the
  \ac{BLAS} library is installed in a central 'default' directory
  (like \lstinline{/usr/local/lib/}), this step is not necessary}:
\begin{lstlisting}[basicstyle=\footnotesize\ttfamily]
export LD_LIBRARY_PATH=$LD_LIBRARY_PATH:path_to_/OmicABELnoMM/libs/lib
\end{lstlisting}
Again, make sure to replace |path_to_| with the proper path on your system.

The actual compilation and installation of \oanomm is done by running
the following three commands, which will be explained in more detail
below:
\begin{itemize}
\item |./configure|
\item |make|
\item |make install|
\end{itemize}
The |./configure| step checks whether all required software like
libraries are in place. If not it will warn or even abort. If one or
more of the required libraries can not be found in a standard location
the correct location can be added as an option to the configure step,
for example to point to a different location of the Boost math library
run:
\begin{lstlisting}
./configure --with-boost-include-path=/path/to/boost/lib
\end{lstlisting}
The names of the options for each library can be found by typing
\begin{lstlisting}
./configure --help
\end{lstlisting}

Apart from the library locations, you can specify the installation
location in the configure step. By default, \oanomm will be installed
in the |/usr/local/| directory\footnote{In more detail: the binaries
  will be installed in \lstinline{/usr/local/bin/}, documentation in
  \lstinline{/usr/local/share/omicabelnomm/doc}.}. To change this,
use the |--prefix| option. For example, to install in a subdirectory
of your home directory run
\begin{lstlisting}
./configure --prefix=/home/yourusername/mytools/OmicABELnoMM
\end{lstlisting}


\section{Installing the development version}
For those of you who want to aid in the development of \oanomm

\subsection{autoconf, autotools}

Make sure you have autoconf/autotools installed
\begin{lstlisting}[escapechar=\%]

sudo apt-get install autoconf
autoreconf -fi
autoconf
%
\end{lstlisting}


\section{Setup a project}
\begin{lstlisting}[escapechar=\%]
#projects location
mkdir GWAS_PROJECT

cd GWAS_PROJECT
%
\end{lstlisting}

\section{Library and program Requirements}


\subsection{Compilers}

You will need the latest gcc compiler for your system for running \oanomm on a single multi-core computer .

\begin{lstlisting}[escapechar=\%]
sudo apt-get install gcc-4.9
%
\end{lstlisting}

For compute-cluster you will need \ac{MPI} support.

\begin{lstlisting}[escapechar=\%]
sudo apt-get install openmpi-bin
sudo apt-get install openmpi-common
sudo apt-get install libopenmpi
sudo apt-get install libopenmpi-dbg
sudo apt-get install libopenmpi-dev
\end{lstlisting}


\section{Source Files}

\begin{lstlisting}[basicstyle=\footnotesize\ttfamily,]
svn checkout svn://svn.r-forge.r-project.org/svnroot/genabel/pkg/OmicABELnoMM/

cd OmicABELnoMM
%
\end{lstlisting}

\section{Compiling}

For compiling the final executable binary use:
\begin{lstlisting}[escapechar=\%]
#in /OmicABELnoMM/
make
%
\end{lstlisting}

For compiling the test binary use:
\begin{lstlisting}[escapechar=\%]

#in /OmicABELnoMM/
make check
%
\end{lstlisting}

%%% Local Variables:
%%% mode: latex
%%% TeX-master: "UserGuide"
%%% End:
